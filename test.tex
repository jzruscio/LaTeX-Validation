% Template for PLoS Computational Biology alkdf
% Version 1.0 January 2009
%
% To compile to pdf, run:
% latex plos.template
% bibtex plos.template
% latex plos.template
% latex plos.template
% dvipdf plos.template

\documentclass[10pt]{article}

% amsmath package, useful for mathematical formulas
\usepackage{amsmath}
% amssymb package, useful for mathematical symbols
\usepackage{amssymb}

% graphicx package, useful for including eps and pdf graphics
% include graphics with the command \includegraphics
%\usepackage{graphicx}

% cite package, to clean up citations in the main text. Do not remove.
\usepackage{cite}

\usepackage{color} 

% Use doublespacing - comment out for single spacing
%\usepackage{setspace} 
%\doublespacing


% Text layout
\topmargin 0.0cm
\oddsidemargin 0.5cm
\evensidemargin 0.5cm
\textwidth 16cm 
\textheight 21cm

% Bold the 'Figure #' in the caption and separate it with a period
% Captions will be left justified
%\usepackage[labelfont=bf,labelsep=period,justification=raggedright]{caption}

% Use the PLoS provided bibtex style
\bibliographystyle{plos2009}

% Remove brackets from numbering in List of References
\makeatletter
\renewcommand{\@biblabel}[1]{\quad#1.}
\makeatother


% Leave date blank
\date{}

\pagestyle{myheadings}
%% ** EDIT HERE **

\usepackage{ae,aecompl}

TESTING
\input{notation-def}

%% ** EDIT HERE **
%% PLEASE INCLUDE ALL MACROS BELOW
\newcommand{\srreactionsir}{\mathbf{R_{rev}}}
\newcommand{\xreversearrows}[2]{\overset{#1}{\underset{#2}{\rightleftharpoons}}}
\newcommand{\reversearrows}{\rightleftharpoons}
\newcommand{\one}{ \bf{text}}
\newcommand{text}{ab \over{ cd}}
\newcommand{\test}{ \boldsymbol{TEST}}
\newcommand{\trial}{b\textsuperscript}
%% END MACROS SECTION

\begin{document}

% Title must be 150 characters or less
\begin{flushleft}
{\Large
\textbf{Temperature Control of Fimbriation Circuit Switch in Uropathogenic \emph{Escherichia coli}: Quantitative Analysis via Automated Model Abstraction}
}
% Insert Author names, affiliations and corresponding author email.
\\
Hiroyuki Kuwahara$^{1}$, 
Chris J.\ Myers$^{2}$, 
Michael S.\ Samoilov$^{3,\ast}$
\\
\bf{1} Ray and Stephanie Lane Center for Computational Biology, Carnegie
Mellon University, Pittsburgh, PA 15213, U.S.A.
\\
\bf{2} Department of Electrical and Computer Engineering, University of Utah, Salt Lake City, UT 84112, U.S.A.
\\
\bf{3} QB3: California Institute for Quantitative Biosciences, University of California, Berkeley, CA 94720, U.S.A.
\\
$\ast$ E-mail: mssamoilov@lbl.gov
\end{flushleft}



% Please keep the abstract between 250 and 300 words
\section*{Abstract}
\section*{Author Summary}
\section*{Introduction}
$\beta$

$$ Display Equation$$

$$ 
Display Equation
$$

$$
\frac{\partial \mathcal{M}}{\partial t}=h(U,V) \mathbf{N}
$$
In order to reduce the number of parameters, we assume that Chargaff's second
parity rule holds, so that
$\pi(\mathrm{A}) = \pi(\mathrm{T})$ and $\pi(\mathrm{G}) = \pi(\mathrm{C})$.
Thus, the mutation model only depends on the GC usage,

where $\mathbf{N}$ is the normal to the surface and $h$ a function of the two morphogens.
The simplest case for $h$ that we have adopted in the following is a linear function of one morphogen:
$$h(U,V)=KV$$
where $K$ is a parameter in $\mathbb{R}$.\\

$\partial_t U $ ($\partial_t V$ 
$\textrm{div}(\mathbf{a}U)$ where $\mathbf{a}(M,t)=\frac{dM}{dt}$

$\sqrt{g_t}$

\section*{Acknowledgments}


%\section*{References}
% The bibtex filename
\bibliography{type1pili_MS_new}


\clearpage

\section*{Figure Legends}
%\begin{figure}[!ht]

\begin{figure}[!ht]
\begin{center}
\includegraphics[width=12.0cm]{fig-4}
\end{center}
\caption{
{\bf Detailed \emph{fim} switch configuration model.} 
An abstracted species, \emph{switch}, captures the switching events.
}
\label{fig:fig-switch-conf-model}
\end{figure}

\section{Not Approved}

\end{document}
